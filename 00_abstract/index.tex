\begin{abstract}
    Multi-FPGA systems have usually only been accessible to individuals who are already proficient in the use of FPGA tools and hardware design. This fact seems unfortunate since the necessity for latency-critical systems has increased in the last few years with larger datasets needing to be processed. Multi-FPGA systems offer the parallelism needed to compute these tasks in a reasonable time, however their accessibility fall short of their demand. SystemNaim is tool which aims to explore the possibility of making multi-FPGA systems more accessible to those not proficient in hardware design. The tool allows the user to program a system in the C90 language, and then through the use of function calls, split the resulting system across multiple FPGAs.
    
    In this paper, we will go through the motivation, design methodology, and implementation of SystemNaim and then evaluate, using both qualitative and quantitative methods, whether it achieves it's intended purpose. We will prove that while the tool creates less optimal hardware than a dedicated system, the time saved by using the tool, potentially weeks, can allow for much more experimentation and larger systems to be implemented in reasonable times. We will also show that programs which have an affinity for parallel execution are most the most appropriate to be implemented using SystemNaim, with examples showing up to a 42\% reduction in latency when run on a multi-FPGA system.
\end{abstract}