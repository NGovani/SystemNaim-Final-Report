\begin{abstract}
    Multi-FPGA systems have usually only been accessible to individuals who are already proficient in the use of FPGA pins and hardware design. This fact seems unfortunate since the necessity for latency-critical systems has increased in the last few years with larger datasets needing to be processed. Multi-FPGA systems offer the parallelism needed to compute these tasks in a reasonable time, however their accessibility fall short of their demand. SystemNaim is tool which aims to explore the possibility of making multi-FPGA systems more accessible to those not proficient in hardware design. In this paper, we will go through the motivation, design methodology, and implementation of the tool and then evaluate, using  both qualitative and quantitative methods, whether it achieves it's intended purpose. TODO put in results of evaluations
\end{abstract}