\chapter{Evaluation}

\section{Introduction}

\begin{itemize}
    \item Talk about why the three sections below are being talked about
\end{itemize}

\section{System Performance}

\begin{itemize}
    \item This section goes over how much overhead the multi-fpga system creates versus a single-fgpa system. In essence the cost of the communication bridge.
    \item This section will show that as the off-chip processing takes longer the overhead is becomes more and more negligible making the multi-FPGA system more comparable to the single-FPGA system in terms of latency.
    \item Add in that the benefit of having a multi-FPGA system is an increased amount of resources and thus is we can gain that benefit with a relatively low cost to performance then the aim has been met.
    \item Disclaimer that this only works for systems where the off-chip function only needs two pieces of data. This is unrealistic since most real systems work on large data sets, however streaming in that much data would be very costly to latency.
    \item Potentially perform a test where two elements of an array are passed into every off-chip function call and calculate how long it takes for the system to finish processing.
\end{itemize}

\section{Channel Performance}

\begin{itemize}
    \item This section goes into detail about how to bandwidth of the channel affects the overall performance. 
    \item In most cases higher speeds means better performance, but it seems that in the case of SPI higher speeds also leads to unreliability. Essentially this is an investigation into the fastest bandwidth possible using SPI
    \item Likely not going to be a long section. I have a suspicion that SPI will work fine up until a single point at which is starts failing 100\% off the time. Maybe add some maths to pad it out.
\end{itemize}

\section{Usability}

\begin{itemize}
    \item This section is a qualitative analysis on how difficult the tool is to use. 
    \item Since there's not enough time to ask others to evaluate the tool, a comparison will be made between using the tool to make a multi-FPGA system, and starting from scratch just implementing hardware in Verilog. While it won't be completely accurate, it should be somewhat realistic as I can take inspiration from how long it took me to create certain aspects such as the interconnect.
    \item Make sure to mention that in a pure hardware implementation you probably wouldn't create a giant state machine like the tool generations, instead you'd create a very focused piece of hardware.
\end{itemize}