\chapter{Conclusion}

To summarize the results of our investigation we will revisit the contributions and aims that were originally outlined in \autoref{sec:intro}. The contributions were as follows:

\begin{itemize}
    \item \textbf{Proved it is possible to automate the process of creating a multi-FPGA system.} \autoref{chp:eval} shows SystemNaim working in 2 different testcases, and \autoref{sec:full_system} shows the design flow when using the tool. As shown in the latter, SystemNaim allows the user to input some source files, from which is then creates all the hardware files necessary for a multi-FPGA system. From \autoref{sec:usability} we know that a large portion of the time taken, when creating a multi-FPGA system, can be attributed to the design of the hardware file, and therefore, since SystemNaim generates these files for you, we believe we can conclude that we achieved this contribution.
    \item \textbf{Proved that the tool which performs such a task, can reduce the hardware proficiency needed by the user to create a multi-FPGA system.} \autoref{sec:usability} explains why the use of HLS in order to assist in the automation of creating a system, can also reduce the hardware proficiency needed by the user of the HLS. Furthermore, \autoref{sec:hls_design} shows that the user does not need to learn any additional programming concepts in order to use the “split” and “split\_fpga” constructs, and thus, the only additional knowledge a user would need to implement a multi-FPGA system when using SystemNaim is an understanding of how 3rd party tools, such as Quartus and NIOS are used.
    \item \textbf{Showed that this tool also reduces the development time of implementing a multi-FPGA system.} \autoref{sec:usability} delves into the challenges an individual would face if they decided to implement a multi-FPGA system using only Verilog and Quartus, it then goes on to give an estimated project timeline for the development of such a system. This is all provided in comparison to a similar system implemented using SystemNaim and an estimate of 6 days is saved choose the latter method. This is of course is only an estimate and is also subjective, but given the vast amount of additional challenges faced when not using SystemNaim, as explained in \autoref{sec:usability}, we can conclude that SystemNaim does reduce development time.
    \item \textbf{Additionally, showed that a system generated by the tool, allows the user to achieve a latency decrease.} \autoref{sec:sys_perf} is where we mainly prove this claim. It can be seen that the multi-FPGA system, tested here, beats the single FPGA system with no parallelism exploited. The section then also detail why a multi-FPGA system might be necessary to achieve a latency decrease if not enough resources are available on a single FPGA. With these two conclusions we can then affirm that multi-FPGA system generated by SystemNaim, does in fact let the user decrease the latency of their system.
\end{itemize}

These contributions are the four main ones of the investigation, however they are not the only achievements that this thesis presents. As mentioned in \autoref{sec:intro}, two additional minor contributions can be gained from SystemNaim.

\begin{itemize}
    \item \textbf{HLS Platform.} The first additional contribution is the HLS tool. \autoref{sec:hls_impl} shows the implementation of the HLS tool, and it's line-to-state methodology. The simplicity of this method means the generated Verilog is predictable, as seen in the figures within that section, and thus it provides a good platform from which to develop further advancements. There is no large amount of overhead required to understand the mechanics at play, rather all that is needed is an understanding of Verilog syntax and some experience with compilers. Therefore, we believe that the HLS tool on its own can be used as either a teaching tool or a platform for future research.
    \item \textbf{Interconnect Modularity} \autoref{sec:interconnect_design} specifies the necessity of the interconnect being modular in design. This feature is then proved to exist in \autoref{sec:impl_interconnect}, which delves into why splitting the interconnect into two submodules allows for modularity. The contribution is important because it improves the versatility of the interconnect module, and allows it to be used with a variety of communication channels. This means that not only does it enable SystemNaim to generate multi-FPGA systems that work with multiple communication protocols, but also allows for the interconnect to be used outside of SystemNaim by those who want to implement their own multi-FPGA systems using something similar to an RPC model. Therefore, we'd like to think that specification behind the interconnect is what is really of value here to the wider hardware development community
\end{itemize}

In conclusion, we think that this investigation has proved successful on all the aims it intended meet. Unfortunately, SystemNaim is not a tool that is intended for public use, since its feature set is far too small to be usable for real world designs, mainly it's lack of arrays. However, it has allowed us to perform a thorough and fair investigation of the problem space, and we think that the field of High-Level Synthesis for multi-FPGA systems has grown as a result of this thesis. 