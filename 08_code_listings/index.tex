\chapter{Code Listings}

\begin{figure}[!h]
    \begin{minipage}{0.45\textwidth}
    \centering
    \begin{minted}{c}
holda = func_a(h, n/2);
holdb = func_b(h, (n/2-1));
    \end{minted}
    \end{minipage}
    \begin{minipage}{0.45\textwidth}
    \centering
    \begin{minted}{c}
split{
    holda = func_a(h, n/2);
    holdb = func_b(h, (n/2-1));
}
    \end{minted}
    \end{minipage}
    \captionof{listing}{How to call functions in parallel in SystemNaim. \textit{Left is original, right is modified code}}
     \label{lst:org_to_par}
\end{figure}

\begin{figure}[!h]
    \begin{minipage}{0.45\textwidth}
    \centering
    \begin{minted}{c}
split{
    holda = func_a(h, n/2);
    holdb = func_b(h, (n/2-1));
}
    \end{minted}
    \end{minipage}
    \begin{minipage}{0.45\textwidth}
    \centering
    \begin{minted}{c}
split_fpga{
    holda = func_a(h, n/2);
    holdb = func_b(h, (n/ -1));
}
    \end{minted}
    \end{minipage}
    \captionof{listing}{How to call functions off-chip and in parallel in SystemNaim. \textit{On the left functions are run parallel on the same FPGA, on the right the first function is called off-chip}}
     \label{lst:par_to_off}
\end{figure}

\begin{listing}
    \inputminted[]{c}{08_code_listings/code/integration_seq.c}
    \caption{C code for Composite Simpson integration. Note, no parallelism is used here. }
    \label{lst:c_simp_st}
\end{listing}


\begin{listing}
    \inputminted[]{c}{08_code_listings/code/host_fpga.c}
    \caption{C code used in NIOS II for Eclipse to run the parent FPGA }
    \label{lst:parent_code}
\end{listing}

\begin{listing}
    \inputminted[]{c}{08_code_listings/code/child_fpga.c}
    \caption{C code used in NIOS II for Eclipse to run the child FPGA }
    \label{lst:child_code}
\end{listing}
